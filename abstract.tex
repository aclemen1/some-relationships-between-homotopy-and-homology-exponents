\documentclass[12pt,a4paper]{article}
\usepackage[english]{babel}

\begin{document}
\sloppy

\title{Some relationships between\\ homotopy and homology exponents}
\author{Alain Cl\'ement}
\maketitle

\begin{abstract}
We investigate the integral homology of spaces having a homotopy exponent.
The main examples considered here are simply connected Postnikov pieces with
finite homotopy groups and we would like to know if it is possible for them
to have a homology exponent.

By the work of \mbox{H. Cartan} it is well known that the answer is "no" for
simply connected \mbox{Eilenberg-MacLane} spaces associated to finite groups
since their integral homology groups contain torsion elements of arbitrarily
high order. Therefore any finite product of such spaces has the same
property.

The situation is less trivial when the \mbox{Postnikov} invariants of a
\mbox{Postnikov} piece are non-trivial. In some cases the space retracts
onto an \mbox{Eilenberg-MacLane} and thus the question has an obvious
\mbox{\it topological}  answer. We shall exhibit more complicated spaces
having elements of arbitrarily high order in integral homology by using
such \mbox{\it algebraic} tools as \mbox{Cartan's work} and the Bockstein
spectral sequence.
\end{abstract}

\end{document}
